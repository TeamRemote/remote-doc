\documentclass[11pt,english]{article}
\usepackage{authblk}
\usepackage{babel}
\usepackage{hyperref}
\usepackage{cleveref}
\date{Monday, April 13th, 2014}
\title{CMPSC440 Final Project Proposal}
\author[1]{Hawk Weisman}
\author[1]{Dibyojyoti Mukherjee}
\author[1]{Andreas Bach Landgrebe}
\author[2]{Soukaina Hamimoune}
\affil[1]{Allegheny College, Department of Computer Science}
\affil[2]{Al Akhawayn University, Department of Computer Science}
\begin{document}
\maketitle
\section{Introduction}
	\textit{Pair programming}, a practice in which two programmers work collaboratively on the same source code, is a major component of the Agile and Extreme Programming software engineering methodologies. Empirical studies have shown that pair programming improves code quality and reduces bugs and defects, as well as resulting in more efficient and effective designs \cite{cockburn2000costs}. However, a major drawback of this practice is that it traditionally requires the collaborating programmers to share the same physical location, which limits the applicability of pair programming for open-source projects, or for companies where programmers are physically distant. In Computer Science 440, we were asked to implement a project involving remote communication, and we chose to design and create a system allowing remote pair programming, allowing two or more programmers to view the same source-code document and edit it in real time.

	SublimeText is a popular and extensible text editor, with a rich plugin API. Plugins are implemented in Python, using special SublimeText Python modules to interact with the editor. Team Remote chose SublimeText as the primary platform for the Remote Collab plugin due to this extensibility, as well as the robust Git-based plugin distribution system known as Package Control. It is our eventual goal to provide Remote Collab plugins for other text editors, such as Vim and Atom, as well. While implementing plugins for multiple text editors exceeds the scope of the Computer Science 440 final project, the Remote Collab system will be designed and implemented in such a way that it should be easy to create plugins for other editors.

For the final project of Computer Science 440, our group has decided to create something under the topic of remote communcation. In order to do this remote communication, we have decided to create a plug-in for Sublime Text 3 in order to communicate between different Sublime Text users. We will be using GitHub to best communicate between eachother in the group. The purpose of this assignment that we have choose is to be able to have different programmers to communicate with eacother easier. As we work on this assignment, we will be working with networking. This Sublime Text 3 has been written using python, our group will be working with networking using the python programming language According to the software engineering process, we have had Dibyo and Hawk act as Surgeon and co-surgeon in order to complete this assignment effectively. Andreas will be working on ... Sou will be working on
\section{Architecture and Design}
\section{Implementation}
	\subsection{Implementation Strategy}
		Team Remote has chosen to pursue an implementation strategy based on the philosophy of Agile Software Development. We will 
	\subsection{Team Remote Collab}
		\begin{itemize}
			\item[Dibyojyoti Mukherjee]{Dibyojyoti will leverage his knowledge of networking to act as the point person, or `surgeon',responsible for the implementation of the Remote Collab server.}
			\item[Hawk Weisman]{Hawk will take responsibility for the primary implementation of the Remote Collab SublimeText plugin}
			\item[Andreas Bach Landgrebe]{}
			\item[Soukaina Hamimoune]{}
		\end{itemize}
	\bibliographystyle{plain}
	\bibliography{remote.bib}
\end{document}
